% -------------------------------------------------
% Document class
\documentclass[a4paper]{article}

% Math settings
\usepackage{amssymb}
\usepackage{amsthm}
\usepackage{amsmath}

% Layout settings
% \usepackage{float}
\usepackage{graphicx}
\usepackage{appendix}
\usepackage{hyperref}

% Subfile support
\usepackage{subfiles}

% -------------------------------------------------
% amsthm usage
% \newtheorem{theorem}{Theorem}[section]
% Theorem:定理。是文章中重要的数学化的论述,一般有严格的数学证明。
% Proposition:可以翻译为命题,经过证明且interesting,但没有Theorem重要,比较常用。
% Lemma:一种比较小的定理,通常lemma的提出是为了来逐步辅助证明Theorem,有时候可以将Theorem拆分成多个小的Lemma来逐步证明,以使得证明的思路更加清晰。很少情况下Lemma会以其自身的形式存在。
% Corollary:推论,由Theorem推出来的结论,通常我们会直接说this is a corollary of Theorem A。
% Property:性质,结果值得一记,但是没有Theorem深刻。
% Claim:陈述,先论述然后会在后面进行论证,可以看作非正式的lemma。
% Note:就是注解。
% Remark:涉及到一些结论,相对而言,Note像是说明,而Remark则是非正式的定理。
% Conjecture:猜测。一个未经证明的论述,但是被认为是真。
% Axiom/Postulate:公理。不需要证明的论述,是所有其他Theorem的基础。