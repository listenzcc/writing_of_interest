% -------------------------------------------------
% Document class
\documentclass[a4paper]{article}

\usepackage{listings}
\usepackage[dvipsnames]{xcolor}
\usepackage{color}

% 用来设置附录中代码的样式

\lstset{
    basicstyle=\tt,
    keywordstyle=\color{purple}\bfseries,
    identifierstyle=\color{brown!80!black},
    commentstyle=\color{gray},
    showstringspaces=false,
    frame=tb,
    flexiblecolumns,
    numbers=left,
    breaklines=true,
}

\lstdefinestyle{Python}{
    language=Python,
    columns=fixed,
    basewidth=0.5em,
}

\lstdefinelanguage{JavaScript}{
    keywords={break, case, catch, continue, debugger, default, delete, do, else, finally, for, function, if, in, instanceof, new, return, switch, this, throw, try, typeof, var, void, while, with},
    morecomment=[l]{//},
    morecomment=[s]{/*}{*/},
    morestring=[b]',
    morestring=[b]",
    sensitive=true
}

\lstdefinestyle{JavaScript}{
    language=JavaScript,
    columns=fixed,
    basewidth=0.5em,
}

\newtheorem{theorem}{Theorem}[section]
\newtheorem{lemma}{Lemma}[section]
\newtheorem{proposition}{Proposition}[section]

\title{Concepts}
\author{listenzcc}

\begin{document}

\maketitle

\abstract
Useful concepts of probability and statistics.

\tableofcontents

\section{Concepts}

\subsection{Law of total probability}

What will come may never be absent in the end.
The difference is only how it will be happening.

Thinking from the \textbf{reason} to the \textbf{result}.

\begin{theorem}
    \label{Theroem: Law of total probability}
    Law of total probability

    For random variables $A$ and $B$, we have
    \begin{equation*}
        P(A) = \sum {P(A|B_i) \cdot P(B_i)}, \forall B_i \in B
    \end{equation*}
\end{theorem}

Thinking \textbf{backwardly}.
If we have already know that $A$ only has one option (noted as $a$), which is also inevitable ($P(A=a) = 1$).

\begin{proposition}
    \label{Proposition: Sum of probability of every options is ONE}
    Sum of probability of every options is ONE

    We have $P(a) = 1$ and $P(a|B_i)=1, \forall B_i \in B$.
    Thus,
    \begin{equation*}
        1 = \sum P(B_i), \forall B_i \in B
    \end{equation*}

    Since $a$ can be independent with choice of $B$, thus the proposition may not affected by the choice of $a$.
\end{proposition}

\end{document}
