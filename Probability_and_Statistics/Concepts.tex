% -------------------------------------------------
% Document class
\documentclass[a4paper]{article}

% -------------------------------------------------
% Packages 
\usepackage{hyperref}
\usepackage{amssymb}
\usepackage{amsthm}
\usepackage{amsmath}
\usepackage{appendix}

% -------------------------------------------------
% Code listing 
\usepackage{listings}
\usepackage[dvipsnames]{xcolor}
\usepackage{color}

% Code styles

\lstset{
    basicstyle=\tt,
    keywordstyle=\color{purple}\bfseries,
    identifierstyle=\color{brown!80!black},
    commentstyle=\color{gray},
    showstringspaces=false,
    frame=tb,
    flexiblecolumns,
    numbers=left,
    breaklines=true,
}

\lstdefinestyle{Python}{
    language=Python,
    columns=fixed,
    basewidth=0.5em,
}

\lstdefinelanguage{JavaScript}{
    keywords={break, case, catch, continue, debugger, default, delete, do, else, finally, for, function, if, in, instanceof, new, return, switch, this, throw, try, typeof, var, void, while, with},
    morecomment=[l]{//},
    morecomment=[s]{/*}{*/},
    morestring=[b]',
    morestring=[b]",
    sensitive=true
}

\lstdefinestyle{JavaScript}{
    language=JavaScript,
    columns=fixed,
    basewidth=0.5em,
}

\newtheorem{theorem}{Theorem}[section]
\newtheorem{lemma}{Lemma}[section]
\newtheorem{proposition}{Proposition}[section]

\title{Concepts}
\author{listenzcc}

\begin{document}

\maketitle

\abstract
Useful concepts of probability and statistics.

\tableofcontents

\section{Laws}

\subsection{Law of total probability}

It is common practice to compute the sum of total probability of all available options.

Thinking \textbf{forwardly}.
Which means starting from the \textbf{reason} to the \textbf{result}.

\begin{theorem}
    \label{Theroem: Law of total probability}
    Law of total probability

    For random variables $A$ and $B$, we have
    \begin{equation*}
        P(A) = \sum {P(A|B_i) \cdot P(B_i)}, \forall B_i \in B
    \end{equation*}

    It is automatically accepted that all the $B_i$s are all separable, and mutually exclusive with each other, which means
    \begin{align*}
        P(B_i, B_j) & = P(B_i) \cdot P(B_j), i \neq j \\
        P(B_i, B_i) & = P(B_i)
    \end{align*}
    It is a prior rule to be accepted, and we will accept it if not specified.
\end{theorem}

Thinking \textbf{backwardly}.
If we have already known that $A$ only has one option (noted as $a$), which is also inevitable ($P(A=a) = 1$).

\begin{proposition}
    \label{Proposition: Sum of probability of every options is ONE}
    Sum of probability of every options is ONE

    \begin{proof}
        We have $P(a) = 1$ and $P(a|B_i)=1, \forall B_i \in B$.
        Thus,
        \begin{equation*}
            1 = \sum P(B_i), \forall B_i \in B
        \end{equation*}

        Since $a$ can be something that naturally happens regardless of the choice of $B$, the proposition may not affected by the choice of $a$.
        Thus, the total probability of all options of $B$ is $1$.
    \end{proof}
\end{proposition}

\section{Definitions}

\subsection{Definition of $e$}

The constant $e$ is defined as the infinity integral of
\begin{equation}
    \label{Equation: Definition of e}
    e = \lim_{x \to \infty} (1+\frac{1}{x})^x
\end{equation}
the very existence of $e$ is based on the fact that integrand function is monotone bounded.

\begin{proposition}
    \label{Proposition: Monotone bounded of e}
    The function $f(x)$ is \textbf{monotone bounded}
    \begin{equation*}
        f(x) = (1 + \frac{1}{x}) ^ x, x \in \mathcal{R}
    \end{equation*}

    \begin{proof}
        Use its discrete sequence version
        \begin{equation*}
            g_n = (1 + \frac{1}{n}) ^ n, n \in \mathcal{N}
        \end{equation*}
        by expanding it, we have
        \begin{align*}
            g_n     & = \sum_{i=0}^n (n, i) \cdot \frac{1}{n^i}                      \\
            g_{n+1} & = \sum_{i=0}^{n+1} (n+1, i) \cdot \frac{1}{(n+1)^i}            \\
            g_{n+1} & = \sum_{i=0}^{n} \frac{(n+1, i)}{n+1} \cdot \frac {1}{n^i} + c
        \end{align*}
        where $c > 0$.
        For every $i$, we have $\frac{(n+1, i)}{n+1} > (n, i) $.
        Thus, the $g_n$ is \textbf{monotone increasing}.

        To find a valid upper bound, we enlarge the $g_n$.
        \begin{equation*}
            g_n < \sum_{i=0}^{n} \frac{1}{i!}
        \end{equation*}
        the enlarge is based on the idea of $(n, i) < \frac{n^i}{i!}$.
        Further,
        \begin{equation*}
            g_n < c + \sum_{i=4}^{n} \frac{1}{2^i}
        \end{equation*}
        where constant $c > 0$.
        It uses the idea of $2^i < i!, i \geq 4$.
        Use the sum of geometric series, we can say the $g_n$ is \textbf{upper bounded}.

        Back to $f(x)$, use \textbf{Squeeze Theorem}
        \begin{equation*}
            (1+\frac{1}{n+1})^n < f(x) < (1+ \frac{1}{n})^{n+1}
        \end{equation*}
        where $n = \lfloor x \rfloor$.
        On the left hand, it equals to $g_{n+1} / (1 + \frac{1}{n+1})$;
        On the right hand, it equals to $g_{n} * (1 + \frac{1}{n})$.
        Based on the analysis above, they are of the same value which is defined as $e$.

        Hence proved.

    \end{proof}

\end{proposition}

\subsection{Gaussian integral}
The gaussian integral is the improper integral defined as
\begin{equation}
    \label{Equation: Gaussian integral}
    I = \int_{-\infty}^{\infty} e^{-x^2} dx
\end{equation}
and the value equals to $\sqrt{\pi}$.

\begin{lemma}
    \label{Lemma: Gaussian integral}
    The gaussian integral equals to $\sqrt{\pi}$
    \begin{equation*}
        \int_{-\infty}^{\infty} e^{-x^2} dx = \sqrt{\pi}
    \end{equation*}

    \begin{proof}
        Use the square-shaped integral
        \begin{equation*}
            I^2 = \int_{-\infty}^{\infty} \int_{-\infty}^{\infty} e^{-x^2} e^{-y^2} dx dy
        \end{equation*}

        Transform into polar coordinates
        \begin{equation*}
            I^2 = \int_{0}^{2\pi} \int_{0}^{\infty} e^{-r^2} r dr d\theta
        \end{equation*}

        Evaluate the integral of $r$
        \begin{equation*}
            \int_{0}^{\infty} e^{-r^2} r dr = \frac{1}{2} \int_{0}^{\infty} e^{-u} du = \frac{1}{2}
        \end{equation*}

        Thus
        \begin{equation*}
            I^2 = \int_{0}^{2\pi} \frac{1}{2} d\theta = \pi
        \end{equation*}

        Hence proved.

    \end{proof}
\end{lemma}

\subsection{Gamma function}
The Gamma function is defined as
\begin{equation}
    \label{Equation: Gamma function}
    \Gamma(z) = \int_0^\infty t^{z-1} e^{-t} dt, Re(z) > 0
\end{equation}

The Gamma function has useful properties
\begin{align*}
    \Gamma(1)   & = 1                                    \\
    \Gamma(z+1) & = z \cdot \Gamma(z)                    \\
    \Gamma(z)   & = 2 \int_0^\infty t^{2z-1} e^{-t^2} dt
\end{align*}

Based on the gaussian integral in \eqref{Equation: Gaussian integral}, we have
\begin{equation*}
    \Gamma(\frac{1}{2}) = \sqrt{\pi}
\end{equation*}

There are countless useful properties of Gamma function.
The followings are from the website \footnote{\url{https://math.wikia.org/wiki/Gamma_function}}.

\subsubsection{Euler's reflection formula}
The Gamma function satisfies the reflection formula
\begin{equation}
    \label{Equation: Euler's reflection formula}
    \Gamma(x) \Gamma(1-x) = \frac{\pi}{sin(x \pi)}
\end{equation}

The proof can be found in \footnote{\url{https://brilliant.org/discussions/thread/proof-of-euler-reflection-formula/}}.
The formula can verify the value of $\Gamma(\frac{1}{2})$.

\subsubsection{Euler's product representation}
The Gamma function can be expressed as an infinite product
\begin{equation}
    \label{Equation: Euler's product representation}
    \Gamma(x) = \frac{1}{x}
    \prod_{n=1}^{\infty} (1 + \frac{1}{n}) ^ {x} (1 + \frac{x}{n})^{-1}
\end{equation}

\subsubsection{Weierstrass's product representation}
The Gamma function can be expressed as another infinite product
\begin{equation}
    \label{Equation: Weierstrass's product representation}
    \Gamma(x) = \frac{exp(-\gamma x)}{x}
    \prod_{n=1}^{\infty} (1+\frac{x}{n})^{-1} exp(\frac{x}{n})
\end{equation}
where $\gamma$ denotes the Euler-Mascheroni constant
\begin{equation}
    \label{Equation: Euler-Mascheroni constant}
    \gamma = \lim_{n \rightarrow \infty}
    \sum_{k=1}^{n} \frac{1}{k} - log(n)
\end{equation}
This definition implies the reflection formula along with the Weierstrass product of sine and is equivalent with Euler's representation when the definition of the Euler-Mascheroni constant is substituted and rearranged.

\subsubsection{Riemann's reflection formula}
The Gamma function plays a key role in the analytic continuation of the Riemann zeta function to the complex plane
\begin{equation}
    \label{Equation: Riemann's reflection formula}
    \zeta (s) = 2^s \pi^{s-1} sin(\frac{\pi s}{2}) \Gamma(1-s) \zeta(1-s)
\end{equation}

This example shows how closely related the Gamma function is to other functions.
Another similar relation with the zeta function is
\begin{equation*}
    \zeta(z) \Gamma(z) = \int_{0}^{\infty}
    \frac{t^{z-1}}{e^t-1} dt
\end{equation*}

\subsubsection{Other properties}
The relationship between Beta function
\begin{equation*}
    \int_{0}^{1} t^{n-1} (1-t)^{m-1} dt =
    \frac{\Gamma(n) \Gamma(m)}{\Gamma(n+m)}
\end{equation*}

Use $t=sin^2{\theta}$ for convenience, we have
\begin{equation*}
    \int_{0}^{\frac{\pi}{2}} cos^{2n-1}(\theta) sin^{2m-1}(\theta) d\theta =
    \frac{\Gamma(n) \Gamma(m)}{2 \Gamma(n+m)}
\end{equation*}

When $n \in \mathcal{N}$, we have
\begin{align*}
    \Gamma(n)           & = (n-1)!                                 \\
    \Gamma(\frac{n}{2}) & = \sqrt{\pi} \frac{(n-2)!!}{2^{(n-1)/2}}
\end{align*}
where $!!$ denotes the operator of double factorial \footnote{\url{https://math.wikia.org/wiki/Double_factorial}}.

\subsection{Sterling's formula}
In short version, the sterling's formula gives an expression of $n!$
\begin{equation}
    \label{Equation: Sterling's formula}
    n! = n^n e^{-n} \sqrt{2 \pi n} [1 + \mathcal{O}(\frac{1}{n})]
\end{equation}

\begin{proof}
    \label{Proof: Sterling's formula}
    Proof the Sterling's formula
    \begin{equation*}
        n! = n^n e^{-n} \sqrt{2 \pi n} [1 + \mathcal{O}(\frac{1}{n})]
    \end{equation*}

    Formulate new formula based on Gamma function
    \begin{equation*}
        f(x) = \frac{\Gamma(x) e^x \sqrt{x}}{x^x} =
        2 \int_{0}^{\infty} e^{x-u^2} (\frac{u}{\sqrt{x}})^{2x-1} du
    \end{equation*}

    Changing variables $u=\sqrt{x}+v$
    \begin{equation*}
        f(x) = 2 \int_{-\sqrt{x}}^{\infty} \phi_x(v) e^{-v^2} dv
    \end{equation*}
    where $\phi_x(v) = e^{-2vx^{1/2}} (1+\frac{v}{\sqrt{x}})^{2x-1}, \forall v \geq -\sqrt{x}$.
    Additionally, we define $\phi_x(v) = 0, \forall v < -\sqrt{x}$, one gets
    \begin{equation*}
        f(x) = 2 \int_{-\infty}^{\infty} \phi_x(v) e^{-v^2} dv
    \end{equation*}

    Firstly, using the series expansion
    \begin{equation*}
        log(1+\frac{v}{\sqrt{x}}) = \frac{v}{\sqrt{x}} - \frac{v^2}{2x} + \cdots
    \end{equation*}
    thus, for a fixed $v$
    \begin{equation*}
        log \phi_x(v) = -v^2 + \mathcal{O}(\frac{1}{\sqrt{x}}), x \rightarrow \infty
    \end{equation*}
    hence
    \begin{align*}
        \phi_x(v)                             & = e^{-v^2 + \mathcal{O}(\frac{1}{\sqrt{x}})} \\
        \lim_{x \rightarrow \infty} \phi_x(v) & = e^{-v^2}
    \end{align*}

    Use Lebesgue dominated convergence theorem
    we conclude
    \begin{equation*}
        lim_{x \rightarrow \infty} \frac{\Gamma(x) e^x \sqrt{x}}{x^x}
        = 2 \int_{-\infty}^{\infty} e^{-2 v^2} dv
        = \sqrt{2 \pi}
    \end{equation*}
    additionally, the approximation can be expressed as
    \begin{equation*}
        \frac{\Gamma(x) e^x \sqrt{x}}{x^x} = \sqrt{2 \pi} \cdot e^{\mathcal{O}(\frac{1}{\sqrt{x}})}
    \end{equation*}

    Use the condition of $n=x, n \in \mathcal{N}$, and $\Gamma(n) = (n-1)!$, we get
    \begin{equation*}
        n! = n^n e^{-n} \sqrt{2 \pi n} [1 + \mathcal{O}(\frac{1}{n})]
    \end{equation*}

    Hence proved.

\end{proof}

\section{Basic concepts}
\subsection{Expectation and Variation}
For a random variable, it follows certain distribution
\begin{equation*}
    x \approx p(x), \forall 0 < p(x) < 1, \int_x p(x) dx = 1
\end{equation*}
its expectation and variation are computed as
\begin{align*}
    \mathcal{E} (x) & = \int_x x \cdot p(x) dx                      \\
    \mathcal{V} (x) & = \int_x (x-\mathcal{E}(x)) ^ 2 \cdot p(x) dx
\end{align*}
additionally, the variation can be computed as
\begin{equation*}
    \mathcal{V} (x) = \mathcal{E} (x^2) - \mathcal{E}^2 (x)
\end{equation*}

\end{document}
