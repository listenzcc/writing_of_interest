% -------------------------------------------------
% Document class
\documentclass[a4paper]{article}

\usepackage{listings}
\usepackage[dvipsnames]{xcolor}
\usepackage{color}

% 用来设置附录中代码的样式

\lstset{
    basicstyle=\tt,
    keywordstyle=\color{purple}\bfseries,
    identifierstyle=\color{brown!80!black},
    commentstyle=\color{gray},
    showstringspaces=false,
    frame=tb,
    flexiblecolumns,
    numbers=left,
    breaklines=true,
}

\lstdefinestyle{Python}{
    language=Python,
    columns=fixed,
    basewidth=0.5em,
}

\lstdefinelanguage{JavaScript}{
    keywords={break, case, catch, continue, debugger, default, delete, do, else, finally, for, function, if, in, instanceof, new, return, switch, this, throw, try, typeof, var, void, while, with},
    morecomment=[l]{//},
    morecomment=[s]{/*}{*/},
    morestring=[b]',
    morestring=[b]",
    sensitive=true
}

\lstdefinestyle{JavaScript}{
    language=JavaScript,
    columns=fixed,
    basewidth=0.5em,
}

\newtheorem{theorem}{Theorem}[section]
\newtheorem{lemma}{Lemma}[section]
\newtheorem{proposition}{Proposition}[section]

\title{Concepts}
\author{listenzcc}

\begin{document}

\maketitle

\abstract
Useful concepts of probability and statistics.

\tableofcontents

\section{Concepts}

\subsection{Law of total probability}

It is common practice to compute the sum of total probability of all available options.

Thinking \textbf{forwardly}.
Which means starting from the \textbf{reason} to the \textbf{result}.

\begin{theorem}
    \label{Theroem: Law of total probability}
    Law of total probability

    For random variables $A$ and $B$, we have
    \begin{equation*}
        P(A) = \sum {P(A|B_i) \cdot P(B_i)}, \forall B_i \in B
    \end{equation*}

    It is automatically accepted that all the $B_i$s are all separable, and mutually exclusive with each other, which means
    \begin{align*}
        P(B_i, B_j) & = P(B_i) \cdot P(B_j), i \neq j \\
        P(B_i, B_i) & = P(B_i)
    \end{align*}
    It is a prior rule to be accepted, and we will accept it if not specified.
\end{theorem}

Thinking \textbf{backwardly}.
If we have already known that $A$ only has one option (noted as $a$), which is also inevitable ($P(A=a) = 1$).

\begin{proposition}
    \label{Proposition: Sum of probability of every options is ONE}
    Sum of probability of every options is ONE

    \begin{proof}
        We have $P(a) = 1$ and $P(a|B_i)=1, \forall B_i \in B$.
        Thus,
        \begin{equation*}
            1 = \sum P(B_i), \forall B_i \in B
        \end{equation*}

        Since $a$ can be something that naturally happens regardless of the choice of $B$, the proposition may not affected by the choice of $a$.
        Thus, the total probability of all options of $B$ is $1$.
    \end{proof}
\end{proposition}

\subsection{Definition of $e$}

The constant $e$ is defined as the infinity integral of
\begin{equation}
    \label{Equation: Definition of e}
    e = \lim_{x \to \infty} (1+\frac{1}{x})^x
\end{equation}
the very existence of $e$ is based on the fact that integrand function is monotone bounded.

\begin{proposition}
    \label{Proposition: Monotone bounded of e}
    The function $f(x)$ is \textbf{monotone bounded}
    \begin{equation*}
        f(x) = (1 + \frac{1}{x}) ^ x, x \in \mathcal{R}
    \end{equation*}

    \begin{proof}
        Use its discrete sequence version
        \begin{equation*}
            g_n = (1 + \frac{1}{n}) ^ n, n \in \mathcal{N}
        \end{equation*}
        by expanding it, we have
        \begin{align*}
            g_n     & = \sum_{i=0}^n (n, i) \cdot \frac{1}{n^i}                      \\
            g_{n+1} & = \sum_{i=0}^{n+1} (n+1, i) \cdot \frac{1}{(n+1)^i}            \\
            g_{n+1} & = \sum_{i=0}^{n} \frac{(n+1, i)}{n+1} \cdot \frac {1}{n^i} + c
        \end{align*}
        where $c > 0$.
        For every $i$, we have $\frac{(n+1, i)}{n+1} > (n, i) $.
        Thus, the $g_n$ is \textbf{monotone increasing}.

        To find a valid upper bound, we enlarge the $g_n$.
        \begin{equation*}
            g_n < \sum_{i=0}^{n} \frac{1}{i!}
        \end{equation*}
        the enlarge is based on the idea of $(n, i) < \frac{n^i}{i!}$.
        Further,
        \begin{equation*}
            g_n < c + \sum_{i=4}^{n} \frac{1}{2^i}
        \end{equation*}
        where constant $c > 0$.
        It uses the idea of $2^i < i!, i \geq 4$.
        Use the sum of geometric series, we can say the $g_n$ is \textbf{upper bounded}.

        Back to $f(x)$, use \textbf{Squeeze Theorem}
        \begin{equation*}
            (1+\frac{1}{n+1})^n < f(x) < (1+ \frac{1}{n})^{n+1}
        \end{equation*}
        where $n = \lfloor x \rfloor$.
        On the left hand, it equals to $g_{n+1} / (1 + \frac{1}{n+1})$;
        On the right hand, it equals to $g_{n} * (1 + \frac{1}{n})$.
        Based on the analysis above, they are of the same value which is defined as $e$.

        Hence proved.

    \end{proof}

\end{proposition}

\subsection{Gamma function}
The Gamma function is defined as
\begin{equation}
    \label{Equation: Gamma function}
    \Gamma(z) = \int_0^\infty t^{z-1} e^{-t} dt, Re(z) > 0
\end{equation}

The Gamma function has useful properties
\begin{align*}
    \Gamma(1)           & = 1                                    \\
    \Gamma(\frac{1}{2}) & = \sqrt{\pi}                           \\
    \Gamma(z)           & = z \cdot \Gamma(z-1)                  \\
    \Gamma(n)           & = n!, n \in \mathcal{N}                \\
    \Gamma(z)           & = 2 \int_0^\infty t^{2z-1} e^{-t^2} dt
\end{align*}
\end{document}
