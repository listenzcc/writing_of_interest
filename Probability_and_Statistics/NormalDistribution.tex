% -------------------------------------------------
% Document class
\documentclass[a4paper]{article}

\usepackage{listings}
\usepackage[dvipsnames]{xcolor}
\usepackage{color}

% 用来设置附录中代码的样式

\lstset{
    basicstyle=\tt,
    keywordstyle=\color{purple}\bfseries,
    identifierstyle=\color{brown!80!black},
    commentstyle=\color{gray},
    showstringspaces=false,
    frame=tb,
    flexiblecolumns,
    numbers=left,
    breaklines=true,
}

\lstdefinestyle{Python}{
    language=Python,
    columns=fixed,
    basewidth=0.5em,
}

\lstdefinelanguage{JavaScript}{
    keywords={break, case, catch, continue, debugger, default, delete, do, else, finally, for, function, if, in, instanceof, new, return, switch, this, throw, try, typeof, var, void, while, with},
    morecomment=[l]{//},
    morecomment=[s]{/*}{*/},
    morestring=[b]',
    morestring=[b]",
    sensitive=true
}

\lstdefinestyle{JavaScript}{
    language=JavaScript,
    columns=fixed,
    basewidth=0.5em,
}

\newtheorem{theorem}{Theorem}[section]
\newtheorem{lemma}{Lemma}[section]
\newtheorem{proposition}{Proposition}[section]

\title{Normal Distribution}
\author{listenzcc}

\begin{document}

\maketitle

\abstract
The hand book of normal distribution.
How it is formulated, what it can predict and why we use it.

\tableofcontents

\section{Normal Distribution in Classic View}

\subsection{Binominal Distribution}
Perform experiment for $n$ times, we assume the trials are independent and follow the same distribution.
The output of the experiment is noted as $1$ or $0$, with no vague.
The probability of observing the $1$ output is noted as $p$.
Then, we have
\begin{equation}
    \label{Equation: Binominal distribution}
    P(n, m) = (n, m) \cdot p^{m} \cdot (1-p)^{n-m}
\end{equation}
where $m$ refers the fact that we observe $1$ output for $m$ times.

It is easy to see that the $P(n, m)$ produces a distribution since
\begin{equation*}
    \sum_{i=0}^{n} P(n, i) = 1, i \in \mathcal{N}
\end{equation*}

Use the computation of \textbf{expectation and Variation} of the random variable, we have
\begin{align*}
    \mathcal{E} (m) & = n \cdot p             \\
    \mathcal{V} (m) & = n \cdot p \cdot (1-p)
\end{align*}

\subsection{Poisson Distribution}
The poisson distribution is the infinity binominal distribution (see \eqref{Equation: Binominal distribution}),
when $p$ is \textbf{small} and $n$ is \textbf{large}.
It is defined as
\begin{equation}
    \label{Equation: Poisson distribution}
    P(k) = \frac{\lambda^k}{k!} \cdot e^{-\lambda}, \lambda=np
\end{equation}

\begin{proof}
    Use the equation of $\lambda = np$, we can rewrite \eqref{Equation: Binominal distribution} as
    \begin{equation*}
        P(n, k) = (n, k) (\frac{\lambda}{n})^k (1-\frac{\lambda}{n})^{n-k}
    \end{equation*}

    Since $p$ is small and $n$ is large, we have $k$ is relatively small compared to $n$.
    As a result, in infinity case,
    \begin{align*}
        \lim_{n \rightarrow \infty} \frac{(n, k)}{n^k}          & = \frac{1}{k!}   \\
        \lim_{n \rightarrow \infty} (1-\frac{\lambda}{n})^{n-k} & = e ^ {-\lambda}
    \end{align*}

    Hence proved.
\end{proof}

In practice, we require $\lambda < 1$ to produce a valid approximation.

To compute the expectation and variation of the poisson distribution, we use the taylor series of Exp function
\begin{equation*}
    e^{\lambda t} = \sum_{k=0}^{\infty} \frac{\lambda^k}{k!}, t_0 = 0
\end{equation*}
it naturally guarantees the property of PDF that
\begin{equation*}
    \sum_{k=0}^{\infty} P(k) = 1
\end{equation*}

Use the definition of the expectation, we have
\begin{align*}
    \mathcal{E} (k) & = \sum_{k=1}^{\infty} \frac{\lambda^k}{(k-1)!} \cdot e^{-\lambda} \\
    \mathcal{E} (k) & = \lambda
\end{align*}

Use the definition of the variation, we have
\begin{align*}
    \mathcal{E} (k^2) & = \sum_{k=1}^{\infty} \frac{k \lambda^k}{(k-1)!} \cdot e^{-\lambda}         \\
    \mathcal{E} (k^2) & = \lambda + \sum_{k=2}^{\infty} \frac{\lambda^k}{(k-2)!} \cdot e^{-\lambda} \\
    \mathcal{E} (k^2) & = \lambda + \lambda^2
\end{align*}
where we used the idea of $k = 1 + (k-1)$.
Thus, the variation is
\begin{equation*}
    \mathcal{V} (k) = \lambda
\end{equation*}

\subsection{Normal Distribution}
When $n$ is large and $p$ is not so small, the poisson distribution fails on approximate the binominal distribution.
The normal distribution is used as a more general replacement.

Basically, when $n$, $np$ and $nq$ are large, the binominal distribution is well approximated by the normal distribution
\begin{equation*}
    p(x) = (n, x) p^x q^{n-x} \approx
    \frac{1}{\sqrt{2 \pi n p q}}
    e^{-(x-np)^2/2npq}
\end{equation*}
where $p+q=1$.
See the website \footnote{\url{http://scipp.ucsc.edu/~haber/ph116C/NormalApprox.pdf}} for detail.

And they are linked based on Sterling's formula,
\begin{equation}
    \label{Equation: Sterling's formula}
    n! = n^n e^{-n} \sqrt{2 \pi n} [1 + \mathcal{O}(1/n)]
\end{equation}
See the website \footnote{\url{https://www.researchgate.net/publication/237571154_A_Very_Short_Proof_of_Stirling's_Formula}} for detail.

Formally, the normal distribution is expressed as
\begin{equation}
    \label{Equation: Normal distribution}
    p(x) = \frac{1}{\sqrt{2 \pi} \sigma} e^{-\frac{(x-\mu)^2}{2 \sigma^2}}
\end{equation}
where expectation $\mu = \lim_{n \rightarrow \infty} np$ and variation $\sigma^2 = \lim_{n \rightarrow \infty} npq$.

\section{Family members}

\section{Examples}

\end{document}